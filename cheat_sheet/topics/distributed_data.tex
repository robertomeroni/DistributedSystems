\section{Concurrency Control}
\subsection{Strict Two-Phase Locking}
\begin{enumerate}
\item When a read or write operation accesses an object within a transaction:
\begin{enumerate}
\item If the object is not already locked, it is locked and the operation proceeds.
\item If the object has a conflicting lock set by another transaction, the transaction must wait until it is unlocked.
\item If the object has a non-conflicting lock set by another transaction, the lock is shared and the operation proceeds.
\item If the object has already been locked in the same transaction, the lock will be promoted if necessary and the operation proceeds (where promotion is prevented by a conflicting lock, rule b is used).
\end{enumerate}

\item When a transaction commits or aborts, the server unlocks all objects it locked for the transaction.
\end{enumerate}

\subsection{Timestamps}
\begin{itemize}
\item Each transaction has a unique timestamp, which:
\begin{itemize}
\item Defines its position in the sequence of transactions.
\item Is assigned when the transaction starts.
\end{itemize}

\item Each object \( x \) maintains the following information:
\begin{enumerate}
\item Write timestamp for its committed value: \( \text{wts}(x) \).
\item Read timestamps, represented by the highest: \( \text{rts}(x) \).
\item Pending read and tentative write operations, along with their corresponding timestamps.
\end{enumerate}

\item Operations are validated upon execution by comparing object and transaction timestamps.
\end{itemize}

\begin{itemize}
\item Transaction \( T \) performs a write operation on \( x \):
\begin{itemize}
\item Valid only if object \( x \) was last read and written by earlier transactions, satisfying the conditions: \( (\text{ts}(T) \geq \text{rts}(x)) \) AND \( (\text{ts}(T) > \text{wts}(x)) \).
\item Add to the tentative writes a new version of the object with a write timestamp set to \( \text{ts}(T) \).
\item Overwrite the version if it already exists.
\item Tentative versions are maintained in order based on their timestamps.
\item If the write operation is not valid, transaction \( T \) is aborted because a transaction with a later timestamp has already read or written the object.
\end{itemize}

\item Transaction \( T \) performs a read operation on \( x \):
\begin{itemize}
\item Valid only if object \( x \) was last written by an earlier transaction, satisfying: \( \text{ts}(T) > \text{wts}(x) \).
\item Must read the version with the highest write timestamp lower than the transaction one.
\item If this is still tentative, the read stays pending until the earlier transaction completes.
\item Otherwise, read the committed version and set \( \text{rts}(x) \) to \( \max\{\text{rts}(x), \text{ts}(T)\} \).
\item If \( T \) has already written the object, this will be used.
\item If the read operation is not valid, transaction \( T \) is aborted.
\end{itemize}

\item Transaction \( T \) performs a commit operation:
\begin{itemize}
\item For each object \( x \) written by \( T \), replace the committed version with the tentative one and set \( \text{wts}(x) \) to \( \text{ts}(T) \).
\item If \( x \) has any pending read, write, or commit operation with a lower timestamp than \( T \), the commit remains pending until those earlier operations complete.
\item Otherwise, the commit proceeds and also resumes any pending read and commit operations waiting for \( T \).
\end{itemize}
\end{itemize}


\subsection{Optimistic Concurrency Control}
\subsubsection{Working phase:}
\begin{itemize}
\item Transaction keeps a tentative version of each object that it updates. This allows the transaction to abort with no effect on the objects.
\item Transaction also keeps the read and write set:
\begin{itemize}
\item Read set records the objects read by the transaction.
\item Write set records the objects written by the transaction.
\end{itemize}
\item Read operations are directed to the tentative version if it exists; otherwise, they are directed to the committed one.
\item Write operations are only performed onto the tentative version.
\end{itemize}

\subsubsection{Validation phase:}
\begin{itemize}
\item A transaction is given a sequence number when entering the validation phase, but real assignment is postponed until successful validation and update.
\item On transaction \( T_{\text{vcommit}} \), check for conflicts with overlapping \( T_i \) transactions (those not yet committed at the start of \( T_{\text{v}} \)).
\item If the transaction is valid, it is allowed to commit; otherwise, abort the transaction or perform some conflict resolution
\end{itemize}

\subsubsection{Update phase:}
Backward validation:
\begin{itemize}
\item Validate a transaction \( T_v \) by comparing its read set with the write set of (committed) transactions \( T_i \) with sequence numbers \( T_{\text{start}} < T_i \leq T_{\text{end}} \).
\begin{itemize}
\item \( T_{\text{start}} \) is the highest sequence number assigned when transaction \( T_v \) enters the working phase.
\item \( T_{\text{end}} \) is the highest sequence number assigned when transaction \( T_v \) enters the validation phase.
\end{itemize}
\item If a conflict is detected, abort the transaction \( T_v \). All write sets of overlapping transactions must be maintained until their last concurrent transaction finishes.
\end{itemize}
Forward validation:
\\
 Validate a transaction \( T_v \) by comparing its write set with the read set of overlapping active (uncommitted) transactions.
\begin{enumerate}
\item Abort the transaction being validated.
\item Abort the conflicting active transactions.
\item Allow the conflicting transactions to complete and then try validation again.
\end{enumerate}

Dynamic read sets must be consistently checked, e.g., block reads of active transactions on contended objects during the validation and update phases.


\subsection{Distributed Commit}
\begin{enumerate}
    \item Coordinator sends a Vote-request to all participants, including itself.
    
    \item Each participant replies with Vote-commit if it can commit locally; otherwise, it replies with Vote-abort and directly aborts.
    
    \item Coordinator collects all the votes, including its own. If everyone voted to commit, it sends Global-commit; otherwise, it sends Global-abort.
    
    \item Participants either COMMIT or ABORT according to the received message and optionally send an ACK when done.
\end{enumerate}

\subsection{Version Vectors}
Each replica maintains a vector for each item, indicating its version number per replica.
    \begin{itemize}
        \item Increment its own version number with every write operation.
        \item Each incoming write includes the vector from a prior read, known as the causal context.
        \item The replica compares both vectors to differentiate between causally-related and concurrent writes.
        \begin{itemize}
            \item Overwrite values from writes that happened before.
            \item Retain values from concurrent writes as siblings.
        \end{itemize}
    \end{itemize}

Dotted version vectors implement this idea:
    \begin{itemize}
        \item Each replica \( k \) maintains a version vector \( VV_k \) for each item, represented as \( \{(r_1, i_1, j_1), \ldots, (r_n, i_n, j_n)\} \).
        \begin{itemize}
            \item Here, \( (r_k, \text{range of events} \{1, i_k\} [, \text{last event} j_k]) \).
            \item For example: \( \{(a, 2), (b, 1), (c, 2, 4)\} \equiv \{a_1, a_2, b_1, c_1, c_2, c_4\} \).
        \end{itemize}
        \item When an incoming write occurs, it includes the vector \( VV_{\text{inc}} \). Otherwise:
        \begin{itemize}
            \item \( VV_{\text{new}} = \text{merge}(VV_{\text{inc}}, VV_k) \)
            \item Increment \( VV_{\text{new}} \).
            \item Obtain the last event of \( VV_{\text{new}} \) as a dot (using causal context from \( VV_{\text{inc}} \)).
            \item Add the new value as a sibling.
            \item Prune all siblings \( s \) for which \( VV_{\text{inc}} \geq VV_k(s) \).
        \end{itemize}
        \item If \( VV_{\text{inc}} \geq VV_k \):
        \begin{itemize}
            \item Increment \( VV_{\text{inc}} \).
            \item Obtain the last event of \( VV_{\text{inc}} \) as a dot.
            \item Write the incoming value as the sole value (pairwise maximum).
        \end{itemize}
    \end{itemize}

\section{quorum-based protocols}

 Clients must receive responses from a quorum of the \( N \) replicas before performing either a read ($N_{R}$)) or a write ($N_{W}$)).
    \begin{itemize}
    \item read-write:
    \(
    N_R + N_W > N
    \)
    
    \item write-write:
    \(
    N_W > \frac{N}{2}
    \)
\end{itemize}

\subsection{Quorum Write}
\begin{enumerate}
    \item{Read phase}
    \begin{itemize}
        \item Send read requests to all (or $N_{R}$) replicas.
        \item Await replies from $N_{R}$ replicas.
        \item Learn the highest version among the replicas.
       \\
       (Merge siblings if needed)   
    \end{itemize}
    
    \item{Write phase}
    \begin{itemize}
        \item Send write requests (containing the version learned) to all (or $N_{W}$) replicas. If $N_{W}$, remaining replicas are updated as a background task (anti-entropy).
        \item Await ACKs from $N_{W}$ replicas.
        \item Write is complete and can return to the user code.
    \end{itemize}
\end{enumerate}

\subsection{Quorum Read}
\begin{enumerate}
    \item{Read phase}
    \begin{itemize}
        \item Send read requests to all (or $N_{R}$) replicas.
        \item Await replies from $N_{R}$ replicas.
        \item If all version numbers are the same, return.
    \end{itemize}
    
    \item{Write phase ('read repair')}
    \begin{itemize}
        \item Otherwise, send write requests (containing the largest version number and the associated up-to-date value) to all (or only outdated) replicas.
        \item Await ACKs from $N_{W}$ (or only updated) replicas.
        \item Return the value to the user code.
    \end{itemize}
\end{enumerate}

\section{Flat Process Groups}
Use process replication to build a flat process group that tolerates process failures:
\begin{itemize}
    \item Processes are identical, and the group response is defined through voting.
\end{itemize}
\subsection{Active Replication}
\begin{itemize}
    \item Clients send requests to all workers and await one reply (crash) or \( K + 1 \) identical replies from workers(Byzantine).
 
    
    \item \( K \)-fault tolerance on worker failure:
    \begin{itemize}
        \item \( K + 1 \) processes can tolerate \( K \) crash/omission failures.
        \item \( 2K + 1 \) processes can tolerate \( K \) Byzantine failures.
    \end{itemize}
(\( K \) failing processes could generate the same wrong reply; hence, \( K + 1 \) correct processes are needed.)
\end{itemize}

\subsection{Quorum-Based}
\begin{itemize}
    \item Clients send requests to all workers and await replies from $N_{R}$ (or $N_{W}$) of them.
    
    \item \( K \)-fault tolerance on worker failure:
    \begin{itemize}
        \item \( 2K + 1 \) and \( 2K \) processes can tolerate \( K \) crash or omission failures for writes and reads, respectively:
    \begin{itemize}
        \item \( N_W = \frac{N}{2} + 1 \); \( N = K + N_W \Rightarrow N > 2K \)
        \item \( N_R + N_W = N + 1 \); \( N = K + N_R \Rightarrow N \geq 2K \)
    \end{itemize}
    
    \item \( 3K + 1 \) and \( 3K \) processes can tolerate \( K \) Byzantine failures for writes and reads, respectively:

    \begin{itemize}
        \item \( N_W = \frac{N + K}{2} + 1 \); \( N = K + N_W \Rightarrow N > 3K \)
        \item \( N_R + N_W = K + N + 1 \); \( N = K + N_R \Rightarrow N \geq 3K \)
    \end{itemize}
    \end{itemize}
\end{itemize}

