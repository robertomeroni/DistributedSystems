\section{Time}
\subsection{Physical Clocks}
Synchronize at least every $R < \delta/2\rho$ to limit skew between two clocks to less than $\delta$ time units
\subsubsection{Cristian Algorithm}
Synchronize nodes with a server using UTC (Coordinated Universal Time) receiver within a specified bound, termed as External synchronization:
\begin{enumerate}
    \item Each client queries the server for time at regular intervals \( R \).
    \item The client adjusts its time to \( TS + \frac{\text{RTT}}{2} \).
    \begin{itemize}
        \item Here, \( \text{RTT} \) denotes the round-trip time, calculated as \( T2 - T1 \).
        \item assumes symmetrical latency.
        \item The accuracy of the client's clock is \( \pm (\frac{\text{RTT}}{2} - \text{Min}) \).
    \end{itemize}
\end{enumerate}
    

\subsubsection{Berkeley Algorithm}
Maintain clock synchronization among entities within a bound (Internal synchronization):
    \begin{enumerate}
        \item Master polls slave clocks at intervals \( R \).\\
       -Master adapts slave clocks considering round-trip times.
        \item Master calculates a fault-tolerant average of slave clocks.\\
        -Discards clocks outside the bound.
        \item Master transmits adjustments to local clocks.
    \end{enumerate}


\subsection{Logical Clocks}
\subsubsection{Lamport}
 Each process \( P_i \) has a local counter \( C_i \).
    \begin{enumerate}
        \item Event at \( P_i \) increments \( C_i \).
        \item \( P_i \) sets \( \text{ts}(m) = C_i \) for message \( m \).
        \item Upon receiving \( m \), \( P_j \) adjusts \( C_j \) and increments.
    \end{enumerate}


\subsubsection{Vector Clocks}
 Each \( P_i \) has \( \text{VC}_i[1 \ldots N] \).
    \begin{enumerate}
        \item \( P_i \) increments \( \text{VC}_i[i] \) and sends \( \text{VC}_i \) with \( m \).
        \item \( P_j \) updates \( \text{VC}_j \) on receiving \( m \).
    \end{enumerate}
 Vector clocks detect causality:
    \begin{enumerate}
        \item \( \text{VC}(a) < \text{VC}(b) \) implies \( a \rightarrow b \).
        \item If neither \( \text{VC}(a) < \text{VC}(b) \) nor \( \text{VC}(b) < \text{VC}(a) \), \( a \parallel b \).
    \end{enumerate}

\section{Chandy-Lamport algorithm}
Assumptions:
\begin{itemize}
    \item Reliable processes and channels.
    \item Unidirectional, FIFO channels.
    \item Strongly connected topology.
    \item Processes execute during snapshot.
    \item Any process initiates a snapshot.
\end{itemize}
Steps by initiator:
\begin{itemize}
    \item Record state.
    \item Send marker on every channel.
    \item Record messages until markers received.
\end{itemize}
On receiving marker over c:
\begin{enumerate}
    \item If no state recorded:
    \begin{enumerate}
        \item Record state.
        \item Send marker on every channel.
        \item Record c as empty set.
        \item Record other channels until markers.
    \end{enumerate}
    \item Otherwise, record state of c as set of messages received over c since it saved its state.
\end{enumerate}
Finish after recording state and all incoming channels.
